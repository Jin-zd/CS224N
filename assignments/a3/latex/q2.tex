\graphicspath{ {images/} }

\titledquestion{Analyzing NMT Systems}[25]

\begin{parts}

    \part[3] Look at the {\monofam{src.vocab}} file for some examples of phrases and words in the source language vocabulary. When encoding an input Mandarin Chinese sequence into ``pieces'' in the vocabulary, the tokenizer maps the sequence to a series of vocabulary items, each consisting of one or more characters (thanks to the {\monofam{sentencepiece}} tokenizer, we can perform this segmentation even when the original text has no white space). Given this information, how could adding a 1D Convolutional layer after the embedding layer and before passing the embeddings into the bidirectional encoder help our NMT system? \textbf{Hint:} each Mandarin Chinese character is either an entire word or a morpheme in a word. Look up the meanings of 电, 脑, and 电脑 separately for an example. The characters 电 (electricity) and  脑 (brain) when combined into the phrase 电脑 mean computer.

    \ifans{
        The convolutional layer can integrate information from neighboring characters in the input sequence. 
        This can help the model learn to recognize and understand the meaning of multi-character words, 
        such as 电脑 (computer), which are common in Mandarin Chinese.
    }


    \part[8] Here we present a series of errors we found in the outputs of our NMT model (which is the same as the one you just trained). For each example of a reference (i.e., `gold') English translation, and NMT (i.e., `model') English translation, please:
    
    \begin{enumerate}
        \item Identify the error in the NMT translation.
        \item Provide possible reason(s) why the model may have made the error (either due to a specific linguistic construct or a specific model limitation).
        \item Describe one possible way we might alter the NMT system to fix the observed error. There are more than one possible fixes for an error. For example, it could be tweaking the size of the hidden layers or changing the attention mechanism.
    \end{enumerate}
    
    Below are the translations that you should analyze as described above. Only analyze the underlined error in each sentence. Rest assured that you don't need to know Mandarin to answer these questions. You just need to know English! If, however, you would like some additional color on the source sentences, feel free to use a resource like \url{https://www.archchinese.com/chinese_english_dictionary.html} to look up words. Feel free to search the training data file to have a better sense of how often certain characters occur.

    \begin{subparts}
        \subpart[2]
        \textbf{Source Sentence:} 贼人其后被警方拘捕及被判处盗窃罪名成立。 \newline
        \textbf{Reference Translation:} \textit{\underline{the culprits were} subsequently arrested and convicted.}\newline
        \textbf{NMT Translation:} \textit{\underline{the culprit was} subsequently arrested and sentenced to theft.}
        
        

        \subpart[2]
        \textbf{Source Sentence}: 几乎已经没有地方容纳这些人,资源已经用尽。\newline
        \textbf{Reference Translation}: \textit{there is almost no space to accommodate these people, and resources have run out.   }\newline
        \textbf{NMT Translation}: \textit{the resources have been exhausted and \underline{resources have been exhausted}.}
        
        

        \subpart[2]
        \textbf{Source Sentence}: 当局已经宣布今天是国殇日。 \newline
        \textbf{Reference Translation}: \textit{authorities have announced \underline{a national mourning today.}}\newline
        \textbf{NMT Translation}: \textit{the administration has announced \underline{today's day.}}
        
        
        
        \subpart[2] 
        \textbf{Source Sentence\footnote{This is a Cantonese sentence! The data used in this assignment comes from GALE Phase 3, which is a compilation of news written in simplified Chinese from various sources scraped from the internet along with their translations. For more details, see \url{https://catalog.ldc.upenn.edu/LDC2017T02}. }:} 俗语有云:``唔做唔错"。\newline
        \textbf{Reference Translation:} \textit{\underline{`` act not, err not "}, so a saying goes.}\newline
        \textbf{NMT Translation:} \textit{as the saying goes, \underline{`` it's not wrong. "}}
        
        
    \end{subparts}
    \ifans{ 
        \\(1)The first one is a singular/plural error. One reason may be that there are not enough examples of plural nouns in the training data.
        To fix this, we could increase the size of the training data which has more examples of plural nouns. \\
        (2)The second one is a repetition error. The model may have repeated the word "resources" due to the lack of diversity in the training data.
        One reason may be that the model is not penalized enough for repeating words and has a high probability of repeating words.
        To fix this, we could enhance the attention mechanism to handle with repeating words. \\
        (3)The third one is a word choice error. The model may have chosen the wrong word due to the lack of context. 
        One reason may be that the model is not able to capture the context of the sentence or has little knowledge with such vocabulary.
        To fix this, we could increase the size of the hidden layers to capture more context. \\
        (4)The fourth one is a translation error. One reason may be that the model concentrates all the attention to the 
        meaning of 'not wrong'. To fix this, we could change the attention calculate style. 
    }


    \part[14] BLEU score is the most commonly used automatic evaluation metric for NMT systems. It is usually calculated across the entire test set, but here we will consider BLEU defined for a single example.\footnote{This definition of sentence-level BLEU score matches the \texttt{sentence\_bleu()} function in the \texttt{nltk} Python package. Note that the NLTK function is sensitive to capitalization. In this question, all text is lowercased, so capitalization is irrelevant. \\ \url{http://www.nltk.org/api/nltk.translate.html\#nltk.translate.bleu_score.sentence_bleu}
    } 
    Suppose we have a source sentence $\bs$, a set of $k$ reference translations $\br_1,\dots,\br_k$, and a candidate translation $\bc$. To compute the BLEU score of $\bc$, we first compute the \textit{modified $n$-gram precision} $p_n$ of $\bc$, for each of $n=1,2,3,4$, where $n$ is the $n$ in \href{https://en.wikipedia.org/wiki/N-gram}{n-gram}:
    \begin{align}
        p_n = \frac{ \displaystyle \sum_{\text{ngram} \in \bc} \min \bigg( \max_{i=1,\dots,k} \text{Count}_{\br_i}(\text{ngram}), \enspace \text{Count}_{\bc}(\text{ngram}) \bigg) }{\displaystyle \sum_{\text{ngram}\in \bc} \text{Count}_{\bc}(\text{ngram})}
    \end{align}
     Here, for each of the $n$-grams that appear in the candidate translation $\bc$, we count the maximum number of times it appears in any one reference translation, capped by the number of times it appears in $\bc$ (this is the numerator). We divide this by the number of $n$-grams in $\bc$ (denominator). \newline 

    Next, we compute the \textit{brevity penalty} BP. Let $len(c)$ be the length of $\bc$ and let $len(r)$ be the length of the reference translation that is closest to $len(c)$ (in the case of two equally-close reference translation lengths, choose $len(r)$ as the shorter one). 
    \begin{align}
        BP = 
        \begin{cases}
            1 & \text{if } len(c) \ge len(r) \\
            \exp \big( 1 - \frac{len(r)}{len(c)} \big) & \text{otherwise}
        \end{cases}
    \end{align}
    Lastly, the BLEU score for candidate $\bc$ with respect to $\br_1,\dots,\br_k$ is:
    \begin{align}
        BLEU = BP \times \exp \Big( \sum_{n=1}^4 \lambda_n \log p_n \Big)
    \end{align}
    where $\lambda_1,\lambda_2,\lambda_3,\lambda_4$ are weights that sum to 1. The $\log$ here is natural log.
    \newline
    \begin{subparts}
        \subpart[5] Please consider this example: \newline
        Source Sentence $\bs$: \textbf{需要有充足和可预测的资源。} 
        \newline
        Reference Translation $\br_1$: \textit{resources have to be sufficient and they have to be predictable}
        \newline
        Reference Translation $\br_2$: \textit{adequate and predictable resources are required}
        
        NMT Translation $\bc_1$: there is a need for adequate and predictable resources
        
        NMT Translation $\bc_2$: resources be sufficient and predictable to
        
        Please compute the BLEU scores for $\bc_1$ and $\bc_2$. Let $\lambda_i=0.5$ for $i\in\{1,2\}$ and $\lambda_i=0$ for $i\in\{3,4\}$ (\textbf{this means we ignore 3-grams and 4-grams}, i.e., don't compute $p_3$ or $p_4$). When computing BLEU scores, show your work (i.e., show your computed values for $p_1$, $p_2$, $len(c)$, $len(r)$ and $BP$). Note that the BLEU scores can be expressed between 0 and 1 or between 0 and 100. The code is using the 0 to 100 scale while in this question we are using the \textbf{0 to 1} scale. Please round your responses to 3 decimal places. 
        \newline
        
        Which of the two NMT translations is considered the better translation according to the BLEU Score? Do you agree that it is the better translation?
        \\
        \ifans{
            For c1:
            \alns{
                & p_1 = \frac{0 + 0 + 0 + 0 + 0 + 1 + 1 + 1 + 1}{9} = \frac{4}{9} \\
                & p_2 = \frac{0 + 0 + 0 + 0 + 0 + 1 + 1 + 1}{8} = \frac{3}{8} \\
                & BP = \exp(1 - \frac{11}{9}) \\
                & BLEU = \exp(- \frac{1}{2} (\log3 + \log 2) - \frac{2}{9}) 
            }
            For c2:
            \alns{
                & p_1 = \frac{1 + 1 + 1 + 1 + 1 + 0}{6} = \frac{5}{6} \\
                & p_2 = \frac{0 + 1 + 1 + 1 + 0}{5} = \frac{3}{5} \\
                & BP = 1 \\
                & BLEU = \exp(- \frac{1}{2} \log 2)
            }
            The BLEU score for c1 is smaller than c2. The BLEU score indicates that c2 is the better translation, but I do not agree with this result.
        }
        
        
        \subpart[5] Our hard drive was corrupted and we lost Reference Translation $\br_1$. Please recompute BLEU scores for $\bc_1$ and $\bc_2$, this time with respect to $\br_2$ only. Which of the two NMT translations now receives the higher BLEU score? Do you agree that it is the better translation?
        \\
        \ifans{
            For c1:
            \alns{
                & p_1 = \frac{0 + 0 + 0 + 0 + 0 + 1 + 1 + 1 + 1}{9} = \frac{4}{9} \\
                & p_2 = \frac{0 + 0 + 0 + 0 + 0 + 1 + 1 + 1}{8} = \frac{3}{8} \\
                & BP = 1 \\
                & BLEU = \exp(- \frac{1}{2} (\log3 + \log 2)) 
            }
            For c2:
            \alns{
                & p_1 = \frac{1 + 0 + 0 + 1 + 1 + 0}{6} = \frac{1}{2} \\
                & p_2 = \frac{0 + 0 + 0 + 1 + 0}{5} = \frac{1}{5} \\
                & BP = 1 \\
                & BLEU = \exp(- \frac{1}{2} (\log 5 + \log 2))
            }
            The BLEU score for c1 is larger than c2. The BLEU score indicates that c1 is the better translation, and I agree with this result.
        }
        
        
        \subpart[2] Due to data availability, NMT systems are often evaluated with respect to only a single reference translation. Please explain (in a few sentences) why this may be problematic. In your explanation, discuss how the BLEU score metric assesses the quality of NMT translations when there are multiple reference transitions versus a single reference translation.
        \\
        \ifans{
            \\
            There is not a perfect translation for a sentence, and different reference translations may have same meanings and it could
            not be told which one is greater. Only using a single reference translation may not be able to cover all the possible translations. 
            When having multiple reference translations, the BLEU score can be calculated by combined all the reference translations with 
            different weights. This can help to evaluate the quality of the translation more accurately.
        }
        
        
        \subpart[2] List two advantages and two disadvantages of BLEU, compared to human evaluation, as an evaluation metric for Machine Translation. 
        \\
        \ifans{
            \\
            Compared to human evaluation, BLEU has the following advantages: \\
            1. BLEU is an automatic evaluation metric, which is fast and easy to calculate. \\
            2. BLEU can be used to compare different models and hyperparameters. \\
            Compared to human evaluation, BLEU has the following disadvantages: \\
            1. BLEU is not perfect, and it may not always correlate well with human judgment. \\
            2. BLEU is based on n-gram precision, which may not capture the meaning of the translation well.
        }
        
        
    \end{subparts}


    \part[4] \emph{Beam search} is often employed to improve the quality of machine translation systems. While you were training the model, beam search results for the same example sentence at different iterations were also recorded in TensorBoard, and accessible in the \emph{TEXT} tab (Fig \ref{fig:beam-search-diagnostics-tensorboard}).

    The recorded diagnostic information includes json documents with the following fields: \texttt{example\_source} (the source sentence tokens), \texttt{example\_target} (the ground truth target sentence tokens), and \texttt{hypotheses} (10 hypotheses corresponding to the search result with beam size 10). Note that a predicted translation is often called \emph{hypothesis} in the neural machine translation jargon.

    \begin{subparts}
        \subpart[2] Did the translation quality improve over the training iterations for the model? Give three examples of translations of the example sentence at iterations 200, 3000, and the last iteration to illustrate your answer. For each iteration, pick the first beam search hypothesis as an example:
        \\
        \ifans{
            \\
            Traget: I was able to provide clarification on some of the matters which were raised at that meeting. \\
            2000: It is not that the United Nations in the United Nations of the United Nations. \\
            6000: I would also like to the Secretary-General of the United Nations. \\
            10000: I would like to say that the Secretary-General has been made by the Secretary-General. \\
            Just as above, the quality of the translation has not been improved obviously.
        }
        
        
        
        \subpart[2] How do various hypotheses resulting from beam search qualitatively compare? Give three other examples of hypotheses proposed by beam search at the last iteration to illustrate your answer.
        \\
        \ifans{
            \\
            From start to end:\\
            I would like to say that the Secretary-General has been made by the Secretary-General. \\
            I would like to express the Secretary-General of the Secretary-General of the Secretary-General. \\
            I would like to thank the Secretary-General of the Secretary-General's report. \\
            As shown above, the three sentences have similar meanings, but the critical words are different.
        }
        
    \end{subparts}



    \begin{figure}
        \centering
        \includegraphics[width=0.7\textwidth]{images/example_translation_beam.jpg}
        \caption{Translation with beam search results for an example sentence are recorded in tensorboard for various iterations. The same data is available in the \texttt{outputs/beam\_search\_diagnostics/} folder in your working directory.}
        \label{fig:beam-search-diagnostics-tensorboard}
    \end{figure}
    

\end{parts}
